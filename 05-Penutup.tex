\chapter{\babLima}

Penelitian ini bertujuan untuk membuat sebuah \textit{framework} MVC yang akan digunakan dalam proses pengembangan SPL berbasis web dengan ABS. Berdasarkan pemaparan pada bab-bab sebelumnya, tujuan dari penelitian ini mencakup beberapa hal yang diantaranya adalah:

\begin{enumerate}
    \item Membuat \textit{web server} sederhana yang dapat digunakan untuk menjalankan SPL berbasis web yang dibuat dengan menggunakan ABS.
    \item Membuat \textit{framework} MVC untuk ABS yang berisi abstraksi untuk menghasilkan halaman ABS, mengakses data, serta \textit{URL routing} yang dapat digunakan oleh para pengembang untuk mengembangkan SPL berbasis web dengan menggunakan ABS.
    \item Membuat \textit{tools} yang dapat digunakan untuk melakukan proses integrasi dan \textit{deployment}.
\end{enumerate}

\noindent
Untuk menyelesaikan permasalahan-permasalahan di atas, maka penulis akan melakukan sebuah penelitian dengan tahapan-tahapan sebagai berikut:

\begin{enumerate}
    \item Melakukan fiksasi studi kasus.
    \item Menganalisan dan memodelkan permasalahan yang ada pada studi kasus tersebut.
    \item Mengimplementasikan hasil analisa dan pemodelan dari studi kasus untuk menghasilkan sebuah \textit{framework} MVC yang utuh.
    \item Mengevaluasi \textit{framework} yang telah berhasil dibuat.
    \item Menuliskan laporan hasil penelitian yang telah dilakukan.
\end{enumerate}